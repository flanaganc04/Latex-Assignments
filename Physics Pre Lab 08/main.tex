\documentclass{article}
\usepackage[utf8]{inputenc}
\usepackage{latexsym,amsfonts,amssymb,amsthm,amsmath}

\setlength{\parindent}{0in}
\setlength{\oddsidemargin}{0in}
\setlength{\textwidth}{6.5in}
\setlength{\textheight}{8.8in}
\setlength{\topmargin}{0in}
\setlength{\headheight}{5pt}

\title{Physics Pre Lab 8 - Conservation of Momentum}
\author{Colin Flanagan}
\date{October 28th, 2024}

\begin{document}

\maketitle

\subsection*{What is the goal of this experiment? (What principles are we studying?)}

    The goal of this experiment is to show that momentum is conserved in elastic collisions and not conserved in inelastic collisions using the carts and sonar range finders.
    

\subsection*{What are some reasons why it is important to study these principles?
}

    Momentum is a concept in physics that allows to solve more complicated problems than forces alone.
    
\subsection*{What basic physics concepts are applicable to this situation, and how do they apply? (Examples: Definitions of movement, Newton’s Laws of Motion, Conservation of Momentum, etc.)}

    Momentum\\

    Conservation of Momentum\\

    Newton's Third Law\\

\subsection*{What are some possible questions you can investigate while performing this lab?
}

   How will the mass of the carts affect the momentum of each cart.\\

   How will the mass affect the velocity of each cart after impact.\\

\subsection*{Describe your experimental design to accomplish the task. (You should not be writing a bulleted list of steps to be completed, but rather a qualitative description of the process which you will refine with your group during lab. Your pre-lab experimental design should be approximately one to three paragraphs in length.)}

    To see the effects of momentum in an elastic situation we will use the magnetic sides of the carts pointed towards each other to repel upon collision. Having one cart in the middle of the track stagnant allows us to set its initial velocity to 0. The carts are also equipped with plungers that allow you to get a consistent initial velocity for the other cart. The range detector can measure the velocity right after the collisions. This information allows us to use the equations for momentum if it is conserved. The equations to investigate this phenomena are 

    \begin{align*}
        \Vec{p}_0 =&  \Vec{p}_f\\
    \end{align*}
    Where $\Vec{p}_0$ is the initial momentum of the carts and $\Vec{p}_f$ is the final momentum of the carts, after the collision. Because there are two carts in the system this needs to be accounted for and then the definition of momentum can be used to verify the conservation.
    \begin{align*} 
        \Vec{p}_0_,_1 + \Vec{p}_0_,_2 =&  \Vec{p}_f_,_1 + \Vec{p}_f_,_2\\
        m_1v_0_,_1 + m_2v_0_,_2 =& m_1v_f_,_1 + m_2v_f_,_2
    \end{align*}

    When investigating the inelastic scenario with the velcro sides of the carts so they stick upon collision is set up the same way but this equation should result in a statement that is untrue. The momentum at the beginning is not equal to the momentum at the end.
\end{document}
