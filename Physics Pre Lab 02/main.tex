\documentclass{article}
\usepackage[utf8]{inputenc}
\usepackage{latexsym,amsfonts,amssymb,amsthm,amsmath}

\setlength{\parindent}{0in}
\setlength{\oddsidemargin}{0in}
\setlength{\textwidth}{6.5in}
\setlength{\textheight}{8.8in}
\setlength{\topmargin}{0in}
\setlength{\headheight}{5pt}

\title{Physics Pre Lab 2}
\author{Colin Flanagan}
\date{September 2024}

\begin{document}

\maketitle

\subsection*{What is the goal of this experiment? (What principles are we studying?)}
    The goal of this experiment is to determine which method of vector addition is the most accurate between Experimental, Graphical, and Component addition.

\subsection*{What are some reasons why it is important to study these principles?
}
    It is important to study these principles because physics describes motion and motion can be represented using vectors. This allows us to add and subtract vectors. 
    
\subsection*{What basic physics concepts are applicable to this situation, and how do they apply? (Examples: Definitions of movement, Newton’s Laws of Motion, Conservation of Momentum, etc.)}
    Gravity, and vectors are the two basic physics concepts that apply to this situation. This is applicable because the masses of the weights are being pulled toward the earth by gravity. Vectors are how we are going to extrapolate information from these forces.

\subsection*{What are some possible questions about vectors you can investigate while performing this lab?
}
    How can we show that the mass needed to balance the table is the opposite of the sum of the other two vectors?
    \\ \\
    How do the two vectors behave when the "opposing" vector is under mass or over mass? How does it affect the other masses?
    \\ \\
    Is tip-tail comparable to parallelogram? Are they the same?

\subsection*{Describe your experimental design to accomplish the task. (You should not be writing a bulleted list of steps to be completed, but rather a qualitative description of the process which you will refine with your group during lab. Your pre-lab experimental design should be approximately one to three paragraphs in length.)}
    To compare the three different methods of vector addition we need to complete all three additions using the same values. In the method Dr. Clark designed he used (70, 0) (100, 120) as the magnitude and direction respectively. These values are arbitrary and just need to be consistent throught out the three experiments. For experimental addition we are using a force table with masses at different angles to represent vectors. For graphical addition we will be using to tip-to-tail method in contrast to the parallelogram method. For component addition the equations are as follows: 
\begin{align}
    F_A_+_B_,_x &= F_A_,_x + F_B_,_x \nonumber\\\nonumber \\
    F_A_+_B_,_y &= F_A_,_y + F_B_,_y \nonumber\\ \nonumber\\
    F_x &= |F|cos\theta \nonumber\\ \nonumber\\ 
    F_y &= |F|sin\theta \nonumber\\ \nonumber\\ 
    |F| &= \sqrt{F_x^2 + F_y^2} \nonumber\\ \nonumber\\ 
    \theta &= tan^-^1(\frac{F_y}{F_x}) \nonumber 
\end{align}
It should be noted that arctan(theta) only returns values between 0 and 90. Refer to table and note (see Lab-Vector Addition.pdf)
 

\end{document}
