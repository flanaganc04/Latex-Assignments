\documentclass{article}
\usepackage[utf8]{inputenc}
\usepackage{latexsym,amsfonts,amssymb,amsthm,amsmath}

\setlength{\parindent}{0in}
\setlength{\oddsidemargin}{0in}
\setlength{\textwidth}{6.5in}
\setlength{\textheight}{8.8in}
\setlength{\topmargin}{0in}
\setlength{\headheight}{5pt}

\title{Physics 204 Pre Lab 8 - Radio}
\author{Colin Flanagan}
\date{March 31st, 2025}

\begin{document}

\maketitle

\subsection*{What is the goal of this experiment? (What principles are we studying?)}

    The goal of this experiment is to create a simple radio transmitter and receiver. 

\subsection*{What are some reasons why it is important to study these principles?
}

  Radios are a common appliance in homes, stores, cars and many other things. It is also important to be able to understand why and how a radio works.
    
\subsection*{What basic physics concepts are applicable to this situation, and how do they apply? (Examples: Definitions of movement, Newton’s Laws of Motion, Conservation of Momentum, etc.)}

    Electric potential\\

    Current\\

    Magnetic field\\

\subsection*{What are some possible questions you can investigate while performing this lab?
}

   How does the shape/size of the antennae/receiver affect the transmission of the radio waves?\\

   How does stacking magnets affect the radio?\\

   How will we define sensitivity relative to a radio in this experiment?\\

\subsection*{Describe your experimental design to accomplish the task. (You should not be writing a bulleted list of steps to be completed, but rather a qualitative description of the process which you will refine with your group during lab. Your pre-lab experimental design should be approximately one to three paragraphs in length.)}

    This week will definitely require some tinkering before jumping into taking measurements. We need to familiarize ourselves with the oscilloscope and function generator. We will also need to discuss our antennae/receiver and how to build them best. 
\end{document}
