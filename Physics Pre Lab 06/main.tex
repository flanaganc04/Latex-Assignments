\documentclass{article}
\usepackage[utf8]{inputenc}
\usepackage{latexsym,amsfonts,amssymb,amsthm,amsmath}

\setlength{\parindent}{0in}
\setlength{\oddsidemargin}{0in}
\setlength{\textwidth}{6.5in}
\setlength{\textheight}{8.8in}
\setlength{\topmargin}{0in}
\setlength{\headheight}{5pt}

\title{Physics Pre Lab 6 - Air Resistance}
\author{Colin Flanagan}
\date{October 15rd, 2024}

\begin{document}

\maketitle

\subsection*{What is the goal of this experiment? (What principles are we studying?)}

    The goal of this experiment is to determine the coefficient of drag for the coffee filters given the air density, gravity, and experimentally determining the mass, velocity, and cross sectional area of the filter.
    

\subsection*{What are some reasons why it is important to study these principles?
}

    Drag is what allows us to fly planes, shoot rockets into space, and torpedos. It is a very important concept for understanding how objects move through fluids.  
    
\subsection*{What basic physics concepts are applicable to this situation, and how do they apply? (Examples: Definitions of movement, Newton’s Laws of Motion, Conservation of Momentum, etc.)}

    Forces and Newton's Laws of Motion\\
    Kinematics

\subsection*{What are some possible questions you can investigate while performing this lab?
}

   How will the mass of the filters affect the coefficient? \\
   
   How can we measure the accelerations and velocity of the filter?\\

   How will the velocity of the object affect the coefficient of drag?\\

\subsection*{Describe your experimental design to accomplish the task. (You should not be writing a bulleted list of steps to be completed, but rather a qualitative description of the process which you will refine with your group during lab. Your pre-lab experimental design should be approximately one to three paragraphs in length.)}

    To measure the coefficient of drag (in this case air resistance) we should examine a free body diagram of the filter falling. It can be seen there are no forces in the x direction. The y direction has the force of gravity pointing down and the force of drag pointing up. This leaves us with the equation

    \begin{align*}
        \Sigma{}F_y = ma_y &= F_d - F_g\\
        ma_y &= \frac{1}{2}C\rho{}Av^2 - mg
    \end{align*}
    With some rearrangements it can be seen the coefficient of drag can be expressed as
    \begin{align*}
        C = \frac{2m(a+g)}{\rho{}Av^2}
    \end{align*}
    So now we now we need to measure the mass of the coffee filter using a scale, the acceleration, the cross sectional area of the filter, and the velocity of the filter. The acceleration due to gravity is given at 9.8$m/s^2$. The density of the air was also given at 1.3$kg/m^3$.\\

    We then need to measure the acceleration of the object by using the sonar detector or by using a timer and a known distance of drop height. The cross sectional area can be measured by measuring the diameter of the coffee filter. Lastly the velocity can be measured using the sonar detector but also using kinematic methods. Both should work.
\end{document}
