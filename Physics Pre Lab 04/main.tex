\documentclass{article}
\usepackage[utf8]{inputenc}
\usepackage{latexsym,amsfonts,amssymb,amsthm,amsmath}

\setlength{\parindent}{0in}
\setlength{\oddsidemargin}{0in}
\setlength{\textwidth}{6.5in}
\setlength{\textheight}{8.8in}
\setlength{\topmargin}{0in}
\setlength{\headheight}{5pt}

\title{Physics Pre Lab 4}
\author{Colin Flanagan}
\date{September 23rd, 2024}

\begin{document}

\maketitle

\subsection*{What is the goal of this experiment? (What principles are we studying?)}

    The goal of this experiment is to determine various properties of a rocket during projectile motion using kinematics and force.
    

\subsection*{What are some reasons why it is important to study these principles?
}

    projectile motion is a very common phenomena that can be studied. A lot can be learned with just a little using math.
    
\subsection*{What basic physics concepts are applicable to this situation, and how do they apply? (Examples: Definitions of movement, Newton’s Laws of Motion, Conservation of Momentum, etc.)}

    Newton's Laws of Motion\\
    Kinematics

\subsection*{What are some possible questions about vectors you can investigate while performing this lab?
}

   How will the launch angle affect the distance traveled? \\
   
   What is the best method to determine the initial velocity?\\

   Can we solve for the force of the thrust?\\

\subsection*{Describe your experimental design to accomplish the task. (You should not be writing a bulleted list of steps to be completed, but rather a qualitative description of the process which you will refine with your group during lab. Your pre-lab experimental design should be approximately one to three paragraphs in length.)}

    We need to launch the rocket at various angles using all of the blocks labeled A-F 30, 35, 40, 45, 50, 55. Launching the rocket what we do know is the acceleration in the y direction which is going to be $9.8 m/s^2$ towards the ground.

    Using the angle created by the block we can turn our acceleration into a vector and manipulate the kinematic equations to solve for the time it took to reach the ground. Similar to solving for the velocity of a baseball being thrown from a height but here we have the full parabola instead of half.

    Using the equation:
    \begin{align*}
        F = ma
    \end{align*}
    We can than find the force of thrust that pushes the rocket of the ground.
 

\end{document}
