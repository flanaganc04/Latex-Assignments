\documentclass{article}
\usepackage[utf8]{inputenc}
\usepackage{latexsym,amsfonts,amssymb,amsthm,amsmath}

\setlength{\parindent}{0in}
\setlength{\oddsidemargin}{0in}
\setlength{\textwidth}{6.5in}
\setlength{\textheight}{8.8in}
\setlength{\topmargin}{0in}
\setlength{\headheight}{5pt}

\title{Physics 204 Pre Lab 4 - Electric Potential Mapping}
\author{Colin Flanagan}
\date{February 16th, 2025}

\begin{document}

\maketitle

\subsection*{What is the goal of this experiment? (What principles are we studying?)}

    The goal of this experiment is to use conductive paper to map electric potential on a 2D plane using a 3D graph.

\subsection*{What are some reasons why it is important to study these principles?
}

  It is important to study electric potential because it is voltage. This is a very important concept for understanding electronics, batteries and much more.
    
\subsection*{What basic physics concepts are applicable to this situation, and how do they apply? (Examples: Definitions of movement, Newton’s Laws of Motion, Conservation of Momentum, etc.)}

    Coulomb's Law\\

    Electric Fields\\

    Electric Potential\\


\subsection*{What are some possible questions you can investigate while performing this lab?
}

   Is there any symmetry, and how can we use that to simplify our math?\\

   Does the electric potential change with the shape of the patterns on the paper.\\


\subsection*{Describe your experimental design to accomplish the task. (You should not be writing a bulleted list of steps to be completed, but rather a qualitative description of the process which you will refine with your group during lab. Your pre-lab experimental design should be approximately one to three paragraphs in length.)}

    The conductive paper has metallic painted points that we can use to create an electric potential difference between the two points. Setting the multimeters negative end to the ground of our circuit, we can then use the positive probe to measure the potential difference at any point on the paper. Measuring these potentials in an x,y coordinate will allow us to map them vertical using the potential reading. The distances will probably start with centimeters but if more precision is need millimeters can be done as well, every half centimeter. The center line/ dipole axis is very important and the potential will be mapped to the equations given in the lab procedure and should align with the $1/r^2$ curve. This is because the electric field strength is based on an inverse square relation with respect to distance. 

\end{document}
