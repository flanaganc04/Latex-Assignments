\documentclass{article}
\usepackage[utf8]{inputenc}
\usepackage{latexsym,amsfonts,amssymb,amsthm,amsmath}

\setlength{\parindent}{0in}
\setlength{\oddsidemargin}{0in}
\setlength{\textwidth}{6.5in}
\setlength{\textheight}{8.8in}
\setlength{\topmargin}{0in}
\setlength{\headheight}{5pt}

\title{Physics 204 Pre Lab 11 - Electrostatic Forces}
\author{Colin Flanagan}
\date{January 26th, 2025}

\begin{document}

\maketitle

\subsection*{What is the goal of this experiment? (What principles are we studying?)}

    The goal of this experiment is to show that the strength of the electric force is inversely proportional to the distance squared between two point charges.
    

\subsection*{What are some reasons why it is important to study these principles?
}

  It is important to study the force of electricity because nearly all the tools we use now have some electronics. This also related to forces and long range forces. This is the second long range force we have investigated. 
    
\subsection*{What basic physics concepts are applicable to this situation, and how do they apply? (Examples: Definitions of movement, Newton’s Laws of Motion, Conservation of Momentum, etc.)}

    Newton's Second Law\\

    Coulomb's Law\\

    Forces\\


\subsection*{What are some possible questions you can investigate while performing this lab?
}

   How does the mass of the hanging object affect the electric force?\\

   How does the length of the string affect the forces? Does it?\\

   How does the charge of the objects affect the force?\\

\subsection*{Describe your experimental design to accomplish the task. (You should not be writing a bulleted list of steps to be completed, but rather a qualitative description of the process which you will refine with your group during lab. Your pre-lab experimental design should be approximately one to three paragraphs in length.)}

    Examining a free body diagram of a small charged sphere being hung from a string and pushed due to electric forces reveals that this sphere has three forces acting on it. The force of tension pulling it in the left and upward direction. The force of gravity pulling it down towards the earth and the force of electricity pushing it to the right. Using newtons second law allows us to set up a simple system of equations for the forces of the ball and see that the force of electricity is equal to 
    \begin{align*}
        F_e = mgtan\theta
    \end{align*}
    where $m$ is the mass of the sphere, $g$ is the acceleration due to gravity and $\theta$ is the angle created between the original string position and the new string position. Setting up the protractor behind the top of the strip the measure the angle and taking the mass of the sphere should be all we need to measure the force at any given distance. To plot distance we will approach the one sphere in 5-10 cm intervals and measuring the angle at each distance. Then using the equation above you can see the relationship between force and distance. 
\end{document}
