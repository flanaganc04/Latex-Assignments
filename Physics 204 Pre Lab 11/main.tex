\documentclass{article}
\usepackage[utf8]{inputenc}
\usepackage{latexsym,amsfonts,amssymb,amsthm,amsmath}

\setlength{\parindent}{0in}
\setlength{\oddsidemargin}{0in}
\setlength{\textwidth}{6.5in}
\setlength{\textheight}{8.8in}
\setlength{\topmargin}{0in}
\setlength{\headheight}{5pt}

\title{Physics 204 Pre Lab 11 - Ray Optics}
\author{Colin Flanagan}
\date{April 28. 2025}

\begin{document}

\maketitle

\subsection*{What is the goal of this experiment? (What principles are we studying?)}

    The goal of this experiment is to measure the speed of light in a rhombic prism, measure the focal length of lenses using images, and another of our choice (colors difference, shadows, or color mixing).

\subsection*{What are some reasons why it is important to study these principles?
}

  Understanding the behavior of light and waves is important for understanding relativity and quantum physics. Rays can helps us simplify the behavior of light and look at more of its properties.
    
\subsection*{What basic physics concepts are applicable to this situation, and how do they apply? (Examples: Definitions of movement, Newton’s Laws of Motion, Conservation of Momentum, etc.)}

    Waves\\
    
    Rays\\

    Lights\\

    Refraction\\

    Refection\\

    Electromagnetic Waves\\

\subsection*{What are some possible questions you can investigate while performing this lab?
}

    How close was our measurement for #1 to the speed of light?\\

    What was our error for each experiment\\

\subsection*{Describe your experimental design to accomplish the task. (You should not be writing a bulleted list of steps to be completed, but rather a qualitative description of the process which you will refine with your group during lab. Your pre-lab experimental design should be approximately one to three paragraphs in length.)}

   For #1 we are going to be using Snell's Law. For #2 we will use a focal point diagram to determine the focal length. Then for number 3 we can use one of these with a different flavor.
\end{document}
