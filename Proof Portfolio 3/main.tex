\documentclass{article}
\usepackage{mathtools}
\usepackage[utf8]{inputenc}
\usepackage{latexsym,amsfonts,amssymb,amsthm,amsmath}
\usepackage{graphicx} % Required for inserting images

\title{Proof Portfolio 3}
\author{Colin Flanagan}
\date{October 4th 2024}

\setlength{\parindent}{0in}
\setlength{\oddsidemargin}{0in}
\setlength{\textwidth}{6.5in}
\setlength{\textheight}{8.8in}
\setlength{\topmargin}{0in}
\setlength{\headheight}{18pt}

\begin{document}

\maketitle

\section*{\underline{Theorem 3.60}} 

\textit{Theorem.} Let $T$ be a right triangle with hypotenuse of length $c$, and sides of length $a$ and $b$. Then $T$ is an isosceles triangle if and only if the area of the right triangle is $\frac{1}{4}c^2$.
    \begin{proof}
     Let $T$ be a right triangle with hypotenuse of length $c$, and sides of length $a$ and $b$. Theorem 3.60 is a biconditional statement and therefore, T is isosceles if the area of the right triangle is $\frac{1}{4}c^2$ and the area of the right triangle T is $\frac{1}{4}c^2$ if T is isosceles.\\
     \\
     \textbf{Case 1.} Suppose T is isosceles. Then it can be seen that side $a$ and side $b$ are equivalent 
     \begin{align*}
         a = b
     \end{align*}
     Because we need to show that the Area = $\frac{1}{4}c^2$ we should examine the formula for the area of this triangle where $b$ denotes the magnitude of the base of the triangle and $h$ denotes the magnitude of the height
     \begin{align*}
         A &= \frac{1}{2}bh\\
         A &= \frac{1}{2}(a)(b)\\
         A &= \frac{1}{2}a(a)\\
         A &= \frac{1}{2}a^2
     \end{align*}
     Consider the Pythagorean theorem of $T$ to isolate $a^2$ and then substitute into the area expression
     \begin{align*}
        a^2 + b^2 &= c^2\\
        a^2 + (a)^2 &= c^2 \\
        2a^2 &= c^2\\
        a^2 &= \frac{c^2}{2}
     \end{align*}
     Now we shall reexamine the expression for the area of $T$ with this new expression for the side $a$
     \begin{align*}
         A &= \frac{1}{2}a^2\\
         A &= \frac{1}{2}(\frac{c^2}{2})\\
         A &= \frac{1}{4}c^2
     \end{align*}
     Therefore if the right triangle $T$ is an isosceles triangle with hypotenuse of length $c$, and sides of length $a$ and $b$ then its area can be written as $\frac{1}{4}c^2$.\\
     \\
     \textbf{Case 2.} Suppose $T$ is a right triangle with hypotenuse of length $c$, and sides of length $a$ and $b$. Suppose the area of $T$ is $\frac{1}{4}c^2$.  Let's consider the two expressions for the area of the triangle, the general expression

     \begin{align*}
         A &= \frac{1}{2}bh\\
         A &= \frac{1}{2}(a)(b)\\
         A &= \frac{1}{2}ab\\
    \end{align*}
    and the expression given,
    \begin{align*}
        A &= \frac{1}{4}c^2.\\ 
    \end{align*}
    We will show these expression are equivalent and use the Pythagorean theorem to substitute for $c^2$
    \begin{align*}
        \frac{1}{2}ab & = \frac{1}{4}c^2\\
        \frac{1}{2}ab & = \frac{1}{4}(a^2+b^2)\\
        2ab &= a^2+b^2.
    \end{align*}
    We will then move $2ab$ to the other side of the expression
    \begin{align*}
        0 &= a^2 - 2ab + b^2\\
        0 &= (a-b)^2\\
        0 &= a-b\\
        b &= a.
    \end{align*}
    If right triangle $T$ has area of $\frac{1}{4}c^2$ then it can be shown that its two sides $a$ and $b$ are equivalent sides. By the definition of an isosceles triangle, $T$ is isosceles. Because both cases of the biconditional statement are true, then $T$ is an isosceles triangle if and only if the area of the right triangle is $\frac{1}{4}c^2$.
    \end{proof}

\end{document}
