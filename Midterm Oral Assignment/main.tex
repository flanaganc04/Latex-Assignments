\documentclass{beamer}

%\usetheme{Boadilla}
\usetheme{Madrid}

\usepackage{latexsym,amsfonts,amssymb,amsthm,amsmath}
\usepackage{verbatim}

\beamertemplatenavigationsymbolsempty

\title{Midterm Oral Assignment}
\author{Colin Flanagan}
\institute{Math 333, Fall 2024}
\date{November 1, 2024}



\begin{document}

\frame{\titlepage}

\frame{
\frametitle{Theorem}

\begin{theorem}
For all integers $r$ and $s$, if $s$ is odd then the equation
$$
x^2 + 2rx + 2s = 0
$$
has no integer solution for x.
\end{theorem}
}


\frame{
\frametitle{Proof Layout}
\begin{itemize}
    \item We will prove this theorem true via contradiction
    \item Employ the theorem of the quadratic formula
    \item Employ the definition of a perfect square
    \item Observe cases of evenness and oddness
\end{itemize}
}

\frame{
\frametitle{Contradiction}
Assume there exists integers $r$ and $s$, if $s$ is odd then the equation
$$
x^2 + 2rx + 2s = 0
$$
does have some integer solutions. If $s$ is odd then 
$$
s= 2p + 1
$$
for some $p \in \Bbb{Z}$.
}

\frame{
\frametitle{Employ Quadratic Formula}
\begin{align*}
    x_1_,_2 &= \frac{-(2r) \pm \sqrt{(2r)^2-4(1)(2(2p+1))}}{2(1)}\\
    \\
    &= \frac{-2r \pm \sqrt{(4r^2-4(2(2p+1))}}{2}\\
    \\
    &= \frac{-2r \pm \sqrt{4(r^2-2(2p+1))}}{2}
\end{align*}
}
\frame{
\frametitle{Algebraic Manipulations}
\begin{align*}
    &= \frac{-2r \pm \sqrt{4(r^2-2(2p+1))}}{2}\\
    \\
    &= \frac{-2r \pm 2\sqrt{r^2-4p-2}}{2}\\
    \\
    x_1_,_2 &= -r \pm \sqrt{r^2-4p-2}
\end{align*}
}

\frame{\frametitle{Adding integers}
For $x_1,_2$ to be a sum and an integer all the elements of that sum need to be an integer
$$
x_1_,_2 &= -r \pm \sqrt{r^2-4p-2}
$$
We know that r is some integer
}

\frame{\frametitle{Definition of a Perfect Square}
The term under the radical must be some perfect square so
$$d^2 = r^2 -4p - 2$$
we will employ the substitution property of addition to add $-r^2$ to both sides
$$d^2 - r^2 = r^2 -  r^2 -4p - 2$$
Employ the additive inverse property
\begin{align*}
    d^2 - r^2 &= 0 -4p - 2\\
    d^2 - r^2 &= -4p - 2
\end{align*}
}

\frame{\frametitle{Factor out the -2}
\begin{align*}
    d^2 - r^2 &= 2(-2p - 1)\\
    d^2 - r^2 &= -2(2p + 1)
\end{align*}
We see that $d^2 - r^2$ is even. We also see it is the product of $-2$ and some odd number. So $d^2 - r^2$ must be the same.
}
\frame{\frametitle{Cases of even and odd}
Square of even number E is even (for some $j \in \Bbb{Z}$)\\
\begin{align*}
    (E)^2 &= (2j)^2\\
    &= 4j^2\\
    &= 2(2j^2)\\
    &= 2t
\end{align*}
Square of odd number D is odd (for some $k \in \Bbb{Z}$)
\begin{align*}
    (D)^2 &= (2k+1)^2\\
    &=4k^2 + 4k + 1\\
    &= 2(2k^2 + 2k) + 1\\
    &=2t + 1
\end{align*}
}
\frame{\frametitle{Even - even, even - odd}
Even - even (for some $j,f \in \Bbb{Z}$)
\begin{align*}
    E_1 - E_2 &= 2j - 2f\\
    &= 2(j-f)\\
    &=2t
\end{align*}
Even - odd (for some $j,k \in \Bbb{Z}$)
\begin{align*}
    E - D &= 2j - 2k - 1\\
    &= 2j - 2k - 2 + 1\\
    &= 2(j-k-1) + 1\\
    &= 2t + 1
\end{align*}
}

\frame{\frametitle{Odd - even, odd - odd}
Odd - even (for some $j,k \in \Bbb{Z}$)
\begin{align*}
    D - E &= 2k + 1 - 2j\\
    &= 2(k-j)+1\\
    &=2t + 1
\end{align*}
Odd - odd (for some $k,f \in \Bbb{Z}$)
\begin{align*}
    D_1 - D_2 &= 2k + 1 - 2f - 1\\
    &= 2k - 2f\\
    &= 2(k-f)\\
    &= 2t
\end{align*}
}

\frame{\frametitle{What does this mean}
We know $d^2 - r^2$ is even so only the even - even case and odd - odd case are possible, now let us make sure the other rule holds true that it is twice the opposite of some odd number. Reexamine these cases:
\begin{align*}
    E_1 - E_2 &= 2j - 2f\\
    &= -2(-j + f)
\end{align*}
\begin{align*}
    D_1 - D_2 &= 2k + 1 - 2f - 1\\
    &= 2k - 2f\\
    &= -2(-k+f)\\
\end{align*}
}
\frame{
\frametitle{References and Further Reading}
Theorem for quadratic formula 
\begin{align*}
    x_1 &= \frac{-b + \sqrt{b^2 - 4ac}}{2a}\\
    \\
     x_2 &= \frac{-b - \sqrt{b^2 - 4ac}}{2a}\\
\end{align*}
}
\end{document}

