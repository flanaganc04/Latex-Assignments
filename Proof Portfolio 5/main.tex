\documentclass{article}
\usepackage{mathtools}
\usepackage[utf8]{inputenc}
\usepackage{latexsym,amsfonts,amssymb,amsthm,amsmath}
\usepackage{graphicx} % Required for inserting images

\title{Proof Portfolio 5}
\author{Colin Flanagan}
\date{November 26th 2024}

\setlength{\parindent}{0in}
\setlength{\oddsidemargin}{0in}
\setlength{\textwidth}{6.5in}
\setlength{\textheight}{8.8in}
\setlength{\topmargin}{0in}
\setlength{\headheight}{18pt}

\begin{document}

\maketitle

\section*{\underline{Theorem}} 

\textit{Theorem.} For all positive integers $n$,
\begin{align*}
    10^n \equiv 1 \pmod{9}.
\end{align*}

\begin{proof}
     We will proceed via induction. Suppose the statement $P(n)$ reads 
 \begin{align*}
     10^n \equiv 1 \pmod{9}
 \end{align*}
 or that 
 \begin{align*}
     9x = 10^n - 1
 \end{align*}
 for some $x \in \Bbb{Z}$ by definitions of divides and congruent/modulo.\\
 
\underline{Base Case:} We wish to show that $P(1)$ is true. This is because 1 is the first element of the set of all positive integers. $P(1)$ states that 
\begin{align*}
    9x &= 10^{(1)} - 1
\end{align*}
Then through manipulation of the right hand side(RHS)
\begin{align*}
    &= 10^{(1)} - 1\\
    &= 10 -1 \\
    &= 9. 
\end{align*}
Through the identity property of multiplication it can be seen that 
\begin{align*}
    &= 9(1)
\end{align*}
the RHS can be written as the product of $9$ and some integer $x$, in the case of $P(1)$ this $x$ is the integer $1$. Therefore, our base case, $P(1)$, true.\\

\underline{Inductive Step:} Let $k\geq 1$ and suppose that $P(k)$ is true, that 
\begin{align*}
    10^k \equiv 1 \pmod{9}
\end{align*}
or that 
\begin{align*}
    9x = 10^k - 1
\end{align*}
for some $x\in\Bbb{Z}$ by the definitions of divides and congruent/modulo.\\
We wish to show that $P(k+1)$ is true, that
\begin{align*}
    10^{k+1} \equiv 1 \pmod{9}
\end{align*}
or, in algebraic terms, that 
\begin{align*}
    9y = 10^{k+1} - 1
\end{align*}
for some $y\in\Bbb{Z}$ by the definitions of divides and congruent/modulo once more. Once again, manipulate the RHS of the equality using rules of exponents
\begin{align*}
    &= 10^{k+1} - 1\\
    &= 10^1(10^k) - 1\\
    &= 10(10^k) - 1\\
\end{align*}
Notice, that to complete the inductive step we must encounter $10^k - 1$ to substitute $P(k)$. One way to do this is by finding a sum that adds to $10$ that involves $1$ so you can distribute. This looks like
\begin{align*}
    &= (9+1)(10^k) - 1\\
    &= 9(10^k) + 1(10^k) - 1\\
    &= 9(10^k) + \underbrace{10^k - 1}.
\end{align*}
Here we can see that we have encountered $P(k)$ and can now substitute to complete the inductive step
\begin{align*}
    &= 9(10^k) + 9x\\
    &= 9(10^k + x)\\
    &= 9z
\end{align*}
Here, the sum of $10^k + 1$ can be written as some integer, $z$, because it is the sum of two integers. Therefore, the RHS is the product of $9$ and some integer $y$, in the case of $P(k+1)$ this $y$ is $z$. Therefore $P(k+1)$ is true, and by the base case, inductive step, and principle of mathematical induction $P(n)$ is true.
    \end{proof}

\end{document}

