\documentclass{article}
\usepackage[utf8]{inputenc}
\usepackage{latexsym,amsfonts,amssymb,amsthm,amsmath}

\setlength{\parindent}{0in}
\setlength{\oddsidemargin}{0in}
\setlength{\textwidth}{6.5in}
\setlength{\textheight}{8.8in}
\setlength{\topmargin}{0in}
\setlength{\headheight}{5pt}

\title{Physics 204 Pre Lab 10 - Single and Double Slit Experiment with a Laser}
\author{Colin Flanagan}
\date{April 19. 2025}

\begin{document}

\maketitle

\subsection*{What is the goal of this experiment? (What principles are we studying?)}

    To measure the wavelenghth of the incident laser through a slit by measuring the minima and maxima refracted light through the slit.

\subsection*{What are some reasons why it is important to study these principles?
}

  Understanding the behavior of light and waves is important for undestanding relativity and quantum physics
    
\subsection*{What basic physics concepts are applicable to this situation, and how do they apply? (Examples: Definitions of movement, Newton’s Laws of Motion, Conservation of Momentum, etc.)}

    Waves\\

    Refraction\\

    Electromagnetic Waves\\

\subsection*{What are some possible questions you can investigate while performing this lab?
}

   How does the laser behave differently for a single slit vs a double slit? A small pin hole? Many holes?\\

   How does the color of the laser change the wavelength\\

   Does adjusting the distance between the laser and the slit or the slit and the screen affect our results/ make it easier to collect to data?

\subsection*{Describe your experimental design to accomplish the task. (You should not be writing a bulleted list of steps to be completed, but rather a qualitative description of the process which you will refine with your group during lab. Your pre-lab experimental design should be approximately one to three paragraphs in length.)}

    The equations being investigated are as follows 
    \begin{align}
        y_p = p\frac{\lambda L}{d}
    \end{align}
    $y_p$ is the distance from each minima to the spot of no refraction, $\lambda$ is the wavelength of the incident light, L is the distance between the slit and the screen, and d is the width of the slit, and p is an integer value representing which minima 1,2,3. Obviously we need to solve for $\lambda$ and the equation 
    \begin{align*}
        \lambda = \frac{y_p d}{p L}
    \end{align*}
    and the equation for the maximas is below and solved for lambda
    \begin{align}
        y_m &= m\frac{\lambda L}{d}\\
        \lambda &= \frac{y_m d}{m L}
    \end{align}
    So to collect our data we need to first set the distance between the slit and the screen, the distances are proportional to the distance between the slit and the screen so a large distance should give us bigger distnaces but maybe less mins and maxs. The distances are inversely proportional to the size of the slit, so a small slit will put the points closer together. I think setting the screen pretty far and then adjusting the size of the hole to get clean mins and maxs is the best bet. Then we need to count which min or max we are measuring the first, the second, the 400th, whichever, then measure the distance from the point of no refraction. 
\end{document}
