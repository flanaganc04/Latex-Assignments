\documentclass{article}
\usepackage[utf8]{inputenc}
\usepackage{latexsym,amsfonts,amssymb,amsthm,amsmath}

\setlength{\parindent}{0in}
\setlength{\oddsidemargin}{0in}
\setlength{\textwidth}{6.5in}
\setlength{\textheight}{8.8in}
\setlength{\topmargin}{0in}
\setlength{\headheight}{5pt}

\title{Physics 204 Pre Lab 6 - Circuits and Resistance}
\author{Colin Flanagan}
\date{March 3rd, 2025}

\begin{document}

\maketitle

\subsection*{What is the goal of this experiment? (What principles are we studying?)}

    The goal of this experiment is to create some simple circuits using a battery pack, wires, and light bulbs and then use those circuits to measure resistance and potential across components of the circuit.

\subsection*{What are some reasons why it is important to study these principles?
}

  Circuitry is used in all modern technologies and most importantly computers. 
    
\subsection*{What basic physics concepts are applicable to this situation, and how do they apply? (Examples: Definitions of movement, Newton’s Laws of Motion, Conservation of Momentum, etc.)}

    Electric Potential\\

    Circuits\\

    Resistance\\

    Ohm's law\\

    Kirchhoff's laws\\
\subsection*{What are some possible questions you can investigate while performing this lab?
}

   How do different configurations of light bulbs affect the resistance from resistor to resistor\\

\subsection*{Describe your experimental design to accomplish the task. (You should not be writing a bulleted list of steps to be completed, but rather a qualitative description of the process which you will refine with your group during lab. Your pre-lab experimental design should be approximately one to three paragraphs in length.)}

    To build circuits that have varying amounts of light bulbs and then measure their resistance using the equation below
    \begin{align*}
        \Delta V = IR
    \end{align*}
    where the change in potential across a component is equal to the current times the resistance. In other words, resistance is proportional to potential change. So we need to use the multimeter to measure the voltage change across the light bulbs and we need to use the multimeter to measure current across the light bulb. Rearranging the above equation, we can then use 
    \begin{align*}
        R = \frac{\Delta V}{I}
    \end{align*}
    to solve for the resistance of the light bulbs.
\end{document}
