\documentclass{article}
\usepackage{mathtools}
\usepackage[utf8]{inputenc}
\usepackage{latexsym,amsfonts,amssymb,amsthm,amsmath}
\usepackage{graphicx} % Required for inserting images

\title{Proof Portfolio 2}
\author{Colin Flanagan}
\date{September 16th 2024}

\setlength{\parindent}{0in}
\setlength{\oddsidemargin}{0in}
\setlength{\textwidth}{6.5in}
\setlength{\textheight}{8.8in}
\setlength{\topmargin}{0in}
\setlength{\headheight}{18pt}

\begin{document}

\maketitle

\section*{\underline{Theorem 3.30} }
\textit{Theorem.} Suppose $a,b \in \mathbb{Z}$. If $a^2(b+3)$ is even then $a$ is even or $b$ is odd.\\

\textit{Proof.} We will proceed via contraposition. If $a$ is odd and $b$ is even then $a^2(b+3)$ is odd. Let \\
\begin{align*}
    a = 2j +1 
\end{align*}
for some $j \in \mathbb{Z}$. Let \\
\begin{align*}
    b = 2k
\end{align*}
for some $k \in \mathbb{Z}$. It can be seen that
\begin{align*}
    a^2(b+3) &= (2j+1)^2((2k)+3)\\
    &= (4j^2+4j+1)(2k+3)\\
    &= 8kj^2 + 12j^2 + 8kj + 12j + 2k + 3\\
    &= 8kj^2 + 12j^2 + 8kj + 12j + 2k + 2 + 1.\\
\end{align*}
Factoring by grouping reveals the equation below,
\begin{align*}
    &= 2(4kj^2 + 6j^2 + 4kj + 6j + k + 1) + 1\\
    &= 2p + 1\\
\end{align*}
where $p$ is some integer $(4kj^2 + 6j^2 + 4kj + 6j + k + 1)$ by the products and sums of integers.Therefore, if $a$ is odd and $b$ is even than $a^2(b+3)$ can be written using the definition of odd. This makes the contrapositive of Theorem 3.30 true making Theorem 3.30 true as well because a statement and its contrapositive are logically equivalent.
\end{document}
