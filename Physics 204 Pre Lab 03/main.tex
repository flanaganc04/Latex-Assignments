\documentclass{article}
\usepackage[utf8]{inputenc}
\usepackage{latexsym,amsfonts,amssymb,amsthm,amsmath}

\setlength{\parindent}{0in}
\setlength{\oddsidemargin}{0in}
\setlength{\textwidth}{6.5in}
\setlength{\textheight}{8.8in}
\setlength{\topmargin}{0in}
\setlength{\headheight}{5pt}

\title{Physics 204 Pre Lab 2 - Electric Field}
\author{Colin Flanagan}
\date{February 3rd, 2025}

\begin{document}

\maketitle

\subsection*{What is the goal of this experiment? (What principles are we studying?)}

    The goal of this experiment is to use conductive paper to measure an electric field in two dimensions using a power supply and a multimeter.
    

\subsection*{What are some reasons why it is important to study these principles?
}

  It is important to study electric fields because they are used in many modern technologies such as circuitry, particle accelerators, but they also have applications in chemistry, like the magnetic fields of atoms and molecules.
    
\subsection*{What basic physics concepts are applicable to this situation, and how do they apply? (Examples: Definitions of movement, Newton’s Laws of Motion, Conservation of Momentum, etc.)}

    Coulomb's Law\\

    Electric Fields\\

    Electric Potential\\


\subsection*{What are some possible questions you can investigate while performing this lab?
}

   Is there any symmetry, and how can we use that to simplify our math?\\

   Does the electric field change with the voltage?\\


\subsection*{Describe your experimental design to accomplish the task. (You should not be writing a bulleted list of steps to be completed, but rather a qualitative description of the process which you will refine with your group during lab. Your pre-lab experimental design should be approximately one to three paragraphs in length.)}

    The conductive paper has metallic painted strips that act as equal potential lines. I am assuming there are two metallic strips so that the positive and negative end of the cables exiting the power supply can both be pinned to a line. From here using the multimeter we can measure the voltage at various points around the paper and record the voltage. The part that is a bit confusing is why we need to touch the black probe to the paper at all, I thought the multimeter could read the volts with just the red probe. However, we need to measure the Volts using the multimeter and come up with a way to measure the distance to use the pre lab statement "For now, we will just accept that the unit V/m = N/C." This way of measuring the electric field isn't really intuitive with the definition
    \begin{align*}
        \vec{E} &= \frac{\vec{F}_{on}_q}{q}\\
        &= \frac{kq_0}{r^2}
    \end{align*}
    so i guess im a little confused but understand the general concept. I'm excited to hear what insight youre going to tell us in lab as well which might point us in a more definitive direction. 

\end{document}
