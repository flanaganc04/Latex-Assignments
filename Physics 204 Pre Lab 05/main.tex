\documentclass{article}
\usepackage[utf8]{inputenc}
\usepackage{latexsym,amsfonts,amssymb,amsthm,amsmath}

\setlength{\parindent}{0in}
\setlength{\oddsidemargin}{0in}
\setlength{\textwidth}{6.5in}
\setlength{\textheight}{8.8in}
\setlength{\topmargin}{0in}
\setlength{\headheight}{5pt}

\title{Physics 204 Pre Lab 5 - Capacitance}
\author{Colin Flanagan}
\date{February 24th, 2025}

\begin{document}

\maketitle

\subsection*{What is the goal of this experiment? (What principles are we studying?)}

    The goal of this experiment is to use household materials to build capacitors and then test their capacitance versus the expected capacitance. 

\subsection*{What are some reasons why it is important to study these principles?
}

  Capacitors are an integral piece of circuitry used in almost all electronics and technologies and understanding how they are made and their function gives insight into improving capacitors.
    
\subsection*{What basic physics concepts are applicable to this situation, and how do they apply? (Examples: Definitions of movement, Newton’s Laws of Motion, Conservation of Momentum, etc.)}

    Coulomb's Law\\

    Electric Fields\\

    Electric Potential\\

    Capacitance
\subsection*{What are some possible questions you can investigate while performing this lab?
}

   Which set of materials makes the best capacitor?\\

   What makes a capacitor the best, its capacitance or applicability?\\

   Which capacitor was farthest from the expected.


\subsection*{Describe your experimental design to accomplish the task. (You should not be writing a bulleted list of steps to be completed, but rather a qualitative description of the process which you will refine with your group during lab. Your pre-lab experimental design should be approximately one to three paragraphs in length.)}

    The expected capacitance for a parallel plate capacitor is 
    \begin{align*}
        C = \kappa\frac{\epsilon_0 A}{d}
    \end{align*}
    which means we need to locate values of kappa for our selected mediums. Other than that building a capacitor is fairly straight forward you need two electrodes and space between them. We will also have to figure out how to use the capacitance meter which I'm assuming you will demo in lab. 
\end{document}
