\documentclass{article}
\usepackage[utf8]{inputenc}
\usepackage{latexsym,amsfonts,amssymb,amsthm,amsmath}

\setlength{\parindent}{0in}
\setlength{\oddsidemargin}{0in}
\setlength{\textwidth}{6.5in}
\setlength{\textheight}{8.8in}
\setlength{\topmargin}{0in}
\setlength{\headheight}{5pt}

\title{Physics Pre Lab 3}
\author{Colin Flanagan}
\date{September 16th, 2024}

\begin{document}

\maketitle

\subsection*{What is the goal of this experiment? (What principles are we studying?)}

    The goal of this experiment is to show that the relationship between force and acceleration are linear. This relationship can also be used to measure the mass of the object.
    

\subsection*{What are some reasons why it is important to study these principles?
}

    This lab relates directly to Newton's Laws of motion. It also brings us into a new concept which is forces.
    
\subsection*{What basic physics concepts are applicable to this situation, and how do they apply? (Examples: Definitions of movement, Newton’s Laws of Motion, Conservation of Momentum, etc.)}

    Newton's Laws of Motion
    Kinematics

\subsection*{What are some possible questions about vectors you can investigate while performing this lab?
}

   How will the mass of the weight attached to the car affect its acceleration as it falls? \\
   
   If there is a difference in the forces caused by the masses is it because of the speed they fall or the mass itself?\\

   How can we determine the uncertainty in the position of the car using the instrumentation provided (range finder)?\\

\subsection*{Describe your experimental design to accomplish the task. (You should not be writing a bulleted list of steps to be completed, but rather a qualitative description of the process which you will refine with your group during lab. Your pre-lab experimental design should be approximately one to three paragraphs in length.)}

    The track will be placed flat on the table with the pulley at the edge of the table. A string is placed over the pulley with the car and mass attached to both ends of the string. When the mass is released it will accelerate the car towards the edge of the table.

    Using the range finder we can find the position and the acceleration of the car. Using the force plate attached to the car we can measured the Force exerted by the car onto our hand when catch it. We can determine the relationship between force and acceleration is linear and dependent upon mass.

    Using the equation:
    \begin{align*}
        F = ma
    \end{align*}
    We can than find the mass of the car.
 

\end{document}
