\documentclass{article}
\usepackage[utf8]{inputenc}
\usepackage{latexsym,amsfonts,amssymb,amsthm,amsmath}

\setlength{\parindent}{0in}
\setlength{\oddsidemargin}{0in}
\setlength{\textwidth}{6.5in}
\setlength{\textheight}{8.8in}
\setlength{\topmargin}{0in}
\setlength{\headheight}{5pt}
\setlength{\parindent}{12pt}

\title{Physics 204 Pre Lab 7 - Magnetic Field and Inductance}
\author{Colin Flanagan}
\date{April 10th, 2025}

\begin{document}

\maketitle

\subsection*{What is the goal of this experiment? (What principles are we studying?)}

    The goal of this experiment is to characterize the direction and magnitude of the magnetic field around a bar magnet, a solenoid with and without current running through it, and the physics lab itself.

\subsection*{What are some reasons why it is important to study these principles?
}

  Magnetic fields allow us to use techniques such as Mass Spectrometry, Nuclear Magnetic Resonance, metal detectors, and many other applications.
    
\subsection*{What basic physics concepts are applicable to this situation, and how do they apply? (Examples: Definitions of movement, Newton’s Laws of Motion, Conservation of Momentum, etc.)}

    Magnetic Field\\

    \noindent Current\\

    \noindent Inductance\\
    
\subsection*{What are some possible questions you can investigate while performing this lab?
}

   How can we characterize the magnetic field inside a solenoid as a function of the variables that go into making it, ie velocity, strength of magnetic field, size of solenoid, number of turns in the solenoid, etc\\

   \noindent How does the velocity of the magnet going through the solenoid affect the current, and how can you quantify that.\\

   \noindent How to use the compasses as a means to measure the magnitude? Maybe the closer they are when they form the circle the weaker the field and vice versa.\\

\subsection*{Describe your experimental design to accomplish the task. (You should not be writing a bulleted list of steps to be completed, but rather a qualitative description of the process which you will refine with your group during lab. Your pre-lab experimental design should be approximately one to three paragraphs in length.)}

    First, I will make my predictions for the bar magnet, solenoid, and room, even though they are not truly predictions.\\
    
        The bar magnet should have a north and south pole. If you put the compass on a given side, the pole of the compass that points to it is the inverse of the pole of the magnet. N-S and S-N. The strength can be determined by moving the magnet closer and farther and seeing when the poles align/ "feel" each other"\\

        Solenoid - should have no net magnetic field because it is not a magnet. Solenoid can be made from ferromagnetic material but are not typically permanent magnet dipoles themselves. However, we know with foresight that the current through the solenoid is dependent on the flux of the magnetic field through the solenoid. Therefore moving a magnet through the solenoid will induce a current that we can measure.\\

        Room - should be aligned with the earth magnetic field, but there may be some slight variation, but overall the compass should roughly point to the same place at all locations indicating the earth has a fairly uniform magnetic field as far as we are concerned on the ground.\\

        Measuring the magnetic field strength with the compass requires finding the minimum distance the compass needs to fully align itself with the field. With the bar magnet this is easy because you can use the poles. The solenoid would be difficult to measure inside, especially if it is small. However, you can use the outside of the solenoid on the principle axis of rotation of the cylinder to find it just outside. The other axes of rotation are not as difficult because you are not constrained by the size of the solenoid to use your compass. For the solenoid with current running through it. We can watch the behavior of the compass as the current changes, as the velocity of the magnet passing in and out changes, the direction, and other characteristics whilst keeping others constant. Doing one piece at a time lets us combine these results into one equation that determines the current given the parameters of the magnet. Measure the room by walking around and moving the magnet, does it stay point in the same direction, is there a sport where it changes.
\end{document}
